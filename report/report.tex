\documentclass{article} % For LaTeX2e
\usepackage{report,times}
\usepackage{hyperref}
\usepackage{url}
\usepackage{biblatex}
\usepackage{amsmath}
\usepackage{graphicx}
\usepackage{algorithmic}
\usepackage{algorithm}
\usepackage{amsfonts}
\usepackage{pgfplots}
\usepackage{pgfplotstable}
\pgfplotsset{compat = 1.17}
\addbibresource{references.bib}
\pgfplotstableread[col sep=comma]{../results/FashionMNIST0.5.csv}\resultsa
\pgfplotstableread[col sep=comma]{../results/FashionMNIST0.6.csv}\resultsb
\pgfplotstableread[col sep=comma]{../results/CIFAR.csv}\resultsc
%\documentstyle[nips14submit_09,times,art10]{article} % For LaTeX 2.09


\title{Assignment 2}
\author{
  John Hu (500395897, zehu4485)
  \and
  Nicholas Grasevski (500710654, ngra5777)
  \and
  Tutor: Yu Yao
}





\newcommand{\fix}{\marginpar{FIX}}
\newcommand{\new}{\marginpar{NEW}}

\nipsfinalcopy % Uncomment for camera-ready version

\begin{document}


\maketitle

\begin{abstract}
The abstract paragraph should be indented 1/2~inch (3~picas) on both left and
right-hand margins. Use 10~point type, with a vertical spacing of 11~points.
The word \textbf{Abstract} must be centered, bold, and in point size 12. Two
line spaces precede the abstract. The abstract must be limited to one
paragraph.
\end{abstract}

\section{Introduction}
In introduction, you should first introduce the problem of learning with label noise, and then its significance and applications. You should give an overview of the methods you want to use.

\section{Related work}
In related work, you are expected to review the main idea of related label noise methods (including their advantages and disadvantages).

\section{Methods}
In methods, you should describe the details of the flip rate estimation methods, include objective function, theoretical foundations (if any), and optimization algorithms. You should also describe the details of your classification models, including the formulation of the cost functions, the theoretical foundations or views (if any) of the cost functions, and the optimization methods.

\section{Experiments}
In experiments, you should introduce your experimental setup (e.g.,datasets, algorithms, evaluation metric, etc.). Then, you should show the experimental results, compare, and analyze your results. If possible, give your personal reflection or thoughts on these results.

\subsection{Datasets}
\subsubsection{Fashion MNIST 0.5}

\subsubsection{Fashion MNIST 0.6}

\subsubsection{CIFAR}

\subsection{Algorithms}
\subsubsection{LeNet-5}

\subsubsection{ResNet-18}

\subsubsection{LightGBM}

\subsection{Evaluation metrics}
\subsubsection{Top-1 Accuracy}
\subsubsection{Relative Reconstruction Errors (RRE)}

\subsection{Results}
\subsubsection{Test accuracy using given transition matrices}
\begin{tikzpicture}
\begin{axis}[
    title={FashionMNIST0.5 - given},
    ybar, nodes near coords,
    xlabel={Model},
    xtick=data,
    xticklabels from table={\resultsa}{model},
    x tick label style={rotate=90,anchor=east},
    ylabel={Top-1 Accuracy (\%)},
]
\addplot table[x expr=\coordindex, y expr=100*\thisrow{acc}]{\resultsa};
\end{axis}
\end{tikzpicture}

\begin{tikzpicture}
\begin{axis}[
    title={FashionMNIST0.6 - given},
    ybar, nodes near coords,
    xlabel={Model},
    xtick=data,
    xticklabels from table={\resultsb}{model},
    x tick label style={rotate=90,anchor=east},
    ylabel={Top-1 Accuracy (\%)},
]
\addplot table[x expr=\coordindex, y expr=100*\thisrow{acc}]{\resultsb};
\end{axis}
\end{tikzpicture}

\begin{tikzpicture}
\begin{axis}[
    title={CIFAR - given},
    ybar, nodes near coords,
    xlabel={Model},
    xtick=data,
    xticklabels from table={\resultsc}{model},
    x tick label style={rotate=90,anchor=east},
    ylabel={Top-1 Accuracy (\%)},
]
\addplot table[x expr=\coordindex, y expr=100*\thisrow{acc}]{\resultsc};
\end{axis}
\end{tikzpicture}


\subsubsection{Test accuracy using estimated transition matrices}
\begin{tikzpicture}
\begin{axis}[
    title={FashionMNIST0.5 - estimated},
    ybar, nodes near coords,
    xlabel={Model},
    xtick=data,
    xticklabels from table={\resultsa}{model},
    x tick label style={rotate=90,anchor=east},
    ylabel={Top-1 Accuracy (\%)},
]
\addplot table[x expr=\coordindex, y expr=100*\thisrow{acc-hat}]{\resultsa};
\end{axis}
\end{tikzpicture}

\begin{tikzpicture}
\begin{axis}[
    title={FashionMNIST0.6 - estimated},
    ybar, nodes near coords,
    xlabel={Model},
    xtick=data,
    xticklabels from table={\resultsb}{model},
    x tick label style={rotate=90,anchor=east},
    ylabel={Top-1 Accuracy (\%)},
]
\addplot table[x expr=\coordindex, y expr=100*\thisrow{acc-hat}]{\resultsb};
\end{axis}
\end{tikzpicture}

\begin{tikzpicture}
\begin{axis}[
    title={CIFAR - estimated},
    ybar, nodes near coords,
    xlabel={Model},
    xtick=data,
    xticklabels from table={\resultsc}{model},
    x tick label style={rotate=90,anchor=east},
    ylabel={Top-1 Accuracy (\%)},
]
\addplot table[x expr=\coordindex, y expr=100*\thisrow{acc-hat}]{\resultsc};
\end{axis}
\end{tikzpicture}


\subsubsection{RRE of estimated transition matrices}
\begin{tikzpicture}
\begin{axis}[
    title={FashionMNIST0.5 - RRE},
    ybar, nodes near coords,
    xlabel={Model},
    xtick=data,
    xticklabels from table={\resultsa}{model},
    x tick label style={rotate=90,anchor=east},
    ylabel={RRE},
]
\addplot table[x expr=\coordindex, y expr=\thisrow{T-hat-RRE}]{\resultsa};
\end{axis}
\end{tikzpicture}

\begin{tikzpicture}
\begin{axis}[
    title={FashionMNIST0.6 - RRE},
    ybar, nodes near coords,
    xlabel={Model},
    xtick=data,
    xticklabels from table={\resultsb}{model},
    x tick label style={rotate=90,anchor=east},
    ylabel={RRE},
]
\addplot table[x expr=\coordindex, y expr=\thisrow{T-hat-RRE}]{\resultsb};
\end{axis}
\end{tikzpicture}

\begin{tikzpicture}
\begin{axis}[
    title={CIFAR - RRE},
    ybar, nodes near coords,
    xlabel={Model},
    xtick=data,
    xticklabels from table={\resultsc}{model},
    x tick label style={rotate=90,anchor=east},
    ylabel={RRE},
]
\addplot table[x expr=\coordindex, y expr=\thisrow{T-hat-RRE}]{\resultsc};
\end{axis}
\end{tikzpicture}

\subsubsection{Estimated transition matrices}


\subsection{Discussion}
\pgfplotstabletypeset[
    columns={model,acc,accestimated,rre},
    columns/model/.style={string type,column name=Model},
    columns/accgiven/.style={string type,column name={Acc\% (given)}},
    columns/accestimated/.style={string type,column name={Acc\% (estimated)}},
    columns/rre/.style={string type,column name={RRE}},
    columns/acc/.style={multiply by=100},
    create on use/accgiven/.style={
        create col/assign/.code={
            \edef\entry{\thisrow{acc}$\pm$\thisrow{acc-std}}
            \pgfkeyslet{/pgfplots/table/create col/next content}\entry
        }
    },
    create on use/accestimated/.style={
        create col/assign/.code={
            \edef\entry{\thisrow{acc-hat}$\pm$\thisrow{acc-hat-std}}
            \pgfkeyslet{/pgfplots/table/create col/next content}\entry
        }
    },
    create on use/rre/.style={
        create col/assign/.code={
            \edef\entry{\thisrow{T-hat-RRE}$\pm$\thisrow{T-hat-RRE-std}}
            \pgfkeyslet{/pgfplots/table/create col/next content}\entry
        }
    },
]{\resultsa}
%\pgfplotstabletypeset{\resultsb}
%\pgfplotstabletypeset{\resultsc}

\section{Conclusion}
In conclusion, you should summarize your methods, results, and your insights for the future work.

\subsection{Future work}


\printbibliography\appendix

\section{Running the code}
The experiment code is contained in a single file \texttt{algorithm.py}, which can either be imported as a module or run as a standalone script.

\begin{itemize}
\item Dependencies: \texttt{pip install jsonlines lightgbm optuna pytorch-lightning}
\item Usage: \texttt{./code/algorithm/algorithm.py >results.csv 2>algorithm.log}
\item Hyperparam tuning: \texttt{./code/algorithm/algorithm.py -t >tuning\_results.jsonl 2>tuning.log}
\item Help: \texttt{./code/algorithm/algorithm.py -h}
\end{itemize}

The code can be configured (number of trials, GPU settings, batch size, etc) via hard-coded constants at the top of the file. At the bottom of the file, there is an additional \texttt{TRAINING\_CONFIG} constant which controls which (model, dataset) combinations to run in the evaluation. Entries can be rearranged and/or commented out if desired.

\end{document}
